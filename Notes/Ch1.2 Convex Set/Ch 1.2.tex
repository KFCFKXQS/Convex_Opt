\documentclass{mytemplate}
\usepackage{hyperref}
\title{Chapter 1.2 凸集}

\author{%
    21307099\\
    \texttt{liyj323@mail2.sysu.edu.cn} \\
}

\begin{document}
\maketitle

\tableofcontents

\section{凸集\ 仿射集\ 凸锥}
\begin{definition}{凸集 Convex Set}\\
    过集合C内任意两点的线段都在C内,则称C为凸集:
    \[x1,x2 \in C \Rightarrow \theta_1 x_1 + \theta_2 x_2 \in C
        , \forall \theta_1 + \theta_2 = 1,\theta_1,\theta_2 \geq 0
    \]
\end{definition}

\begin{definition}{凸组合 Convex Combination}\\
    $\theta_i \geq 0 \ and\  \sum_{i=1}^{m} \theta_i = 1$则
    \text{称$\sum_{i=1}^{m} \theta_i x_i$为$x_1,\dots ,x_m$的一个凸组合}
\end{definition}

\begin{definition}{仿射集 Affine Set}\\
    过集合C内任意两点的直线都在C内,则称C为仿射集:
    \[x1,x2 \in C \Rightarrow \theta_1 x_1 + \theta_2 x_2 \in C
        , \forall \theta_1 + \theta_2 = 1
    \]
\end{definition}
显然仿射集的要求更高,因此任何仿射集都是凸集,但凸集未必是仿射集
\begin{definition}{仿射组合 Affine Combination}\\
    $\sum_{i=1}^{m} \theta_i = 1$则
    \text{称$\sum_{i=1}^{m} \theta_i x_i$为$x_1,\dots ,x_m$的一个仿射组合}
\end{definition}

\fbox{
    \parbox{\textwidth}{
        \subsection{仿射维数和相对内部}
        \begin{definition}{仿射维数}\\
            集合C的仿射维数为其仿射包的维数。
        \end{definition}

        \begin{definition}{相对内部}\\
            定义集合C的相对内部为$\text{aff}\ C$的内部,记为$\text{relint}\ C$即
            \[
                \text{relint}\ C = \{
                x \in C | B(x, r) \cap \text{aff}\ C \subseteq C, \exists r > 0
                \}
            \]
        \end{definition}
        \begin{definition}{相对边界}\\
            定义集合C的相对边界为$\text{cl}C  \text{relint}C$
        \end{definition}
    }
}

\newpage

\begin{definition}{凸锥 Convex Cone}\\
    \[
        \forall x\in C, \theta \geq 0,\text{都有}\theta x \in C, \text{则称C为锥\\}
    \]
    \text{若C为凸集,则称C为凸锥},即
    \[
        x_1,x_2 \in C \Rightarrow \theta_1 x_1 +\theta_2 x_2 \in C, \forall \theta_1, \theta_2 \geq 0
    \]
\end{definition}

\begin{definition}{凸锥组合 Conic Combination}
    \[\theta_1 \dots \theta_m \geq 0\]
    称
    \[
        \sum_{i=1}^{m} \theta_i x_i
    \]
    为$x_1 \dots x_m \geq 0$的一个凸锥组合
\end{definition}

\begin{definition}{凸包\ 仿射包\ 凸锥包}\\
    集合C中任意元素的凸组合、仿射组合、凸锥组合称为C的凸包、仿射包、凸锥包
\end{definition}

\newpage
\section{重要的例子}
\subsection{超平面与半空间}
\paragraph*{超平面:}
\[
    \{x \in \R^n | a^T x=b\}
\]
或者
\[
    \{x \in \R^n | a^T (x-x_0) =0\}
\]
其中$x_0$是超平面上任意一点。超平面是凸集、仿射集,过原点时为凸锥

\paragraph*{半空间:}
\[
    \{x \in \R^n | a^T x \leq b\} , a \neq 0
\]
显然,半空间是凸集,但不是仿射集.过原点时为凸锥

\begin{tikzpicture}
    % 定义坐标轴
    \draw[thick,->] (-1,0) -- (4,0) node[anchor=north west] {$x_1$};
    \draw[thick,->] (0,-1) -- (0,4) node[anchor=south east] {$x_2$};

    % 绘制直线
    \draw[thick] (-0.5,3.75) -- (3,-1.5) node[anchor=west] {$a^Tx = b$ (超平面)};

    % 标记上方区域
    \draw[red,->] (2,1) -- (2.5,1.25) node[anchor=south] {$a^Tx \geq b$ (半空间)};

    % 标记下方区域
    \draw[red,->] (1,0.5) -- (0.5,0.25) node[anchor=north] {$a^Tx \leq b$ (半空间)};

    % 标记原点
    \node at (0,0) [anchor=north east] {0};
\end{tikzpicture}

\newpage
\subsection{范数球}

\begin{definition}{Norm Ball}
    \[\{x \in \R^n | \|x - x_c\|_p \leq r\}
    \]
\end{definition}
显然,范数球是凸集且当$r\neq0$时不可能是仿射集和凸锥

\subsection{球和椭球}
\begin{definition}{球}
    \begin{align}
        b(x_c, r) & =\{x\ |\ \|x-x_c\|_2 \leq r\}             \\
                  & =\{x\ |\ \sqrt{(x-x_c)^T (x-x_c)}\leq r\}
    \end{align}
\end{definition}
\begin{definition}{椭球}\\
    $\mathcal{E}=\{x \in \R^n | (x - x_c)^T P^{-1} (x-x_c) \leq 1\}$ \\
    其中$P=P^T \succ 0$, 半轴长度由$\sqrt{\lambda_i}$给出\\
    \\
    \emph{例如:\[
        x^T\begin{bmatrix}
            4 & 0 \\
            0 & 1
        \end{bmatrix}^{-1}x \leq 1
    \]
    则\[
        \frac{1}{4}x_1^2 + x_2^2 \leq 1
    \]
    特征值为4和1,则半轴长为2和1}
\end{definition}

\subsection{多面体与单纯形}
\begin{definition}{多面体polyhedron}
    \[
        P = \{
        x\ | a_i^T x \leq b_i,\ i = 1\dots m\ \ and\ \
        c_j^T x = d_j,\ j=1\dots p
        \}
    \]
\end{definition}
\ \ \ \ 仿射集合、射线、线段和半空间都是多面体

\begin{definition}{单纯形simplex}\\
    \(v_0,v_1, \dots v_k\)共k+1个点仿射无关(即
    \(v_1-v_0,\dots v_k-k_0\)线性无关),则$C=\text{Conv}\{v_0,v_1,\dots v_k\}$
\end{definition}

\newpage
\subsection{矩阵空间}
\begin{proposition}
    \hspace*{1em}
    \begin{enumerate}
        \item 对称矩阵集$S^n$ \qquad\qquad\quad 显然是凸锥、凸集、仿射集
        \item 对称半正定矩阵集$S_+^n$ \qquad 是凸锥、凸集,不是仿射集(如n=1时为全体非负实数$\R_+$)
        \item 对称正定矩阵集$S_{++}^n$ \qquad 是凸集,不是凸锥、仿射集(如n=1时为全体正实数$R_{++}$)
    \end{enumerate}
\end{proposition}
对称半正定矩阵是凸锥\\
Prove:
\begin{align}
     &                                   &        & \forall A,B \in S_+^n, \forall  x\in \R^n, x^T A x \geq 0, x^T B x \geq 0 &  & \\
     & \forall \theta_0,\theta_1 \geq 0, &        & x^T(\theta_0 A+\theta_1B)x                                                &  & \\
     &                                   & =\     & \theta_0x^T A x+\theta_1x^T B x                                           &  & \\
     &                                   & \geq\  & 0                                                                         &  &
\end{align}
显然$\theta_0, \theta_1$可以同时0,因此严格的>是不满足的,即\textcolor{red}{对称正定矩阵集合不是凸锥}\\
(但显然是凸集,因为凸集中$1^T\vec{\theta} = 1$,不能同时取0)
                       
\section{保凸运算}

\subsection{交集} 
\noindent
交集是保凸的。\\
每一个闭的凸集S都是半空间的交集(通常为无限多个)。事实上,一个闭凸集S是包含它的所有半空间的交集。
(P32原文:\emph{一个闭集S是包含它的所有半空间的交集,但由交集的保凸性可知必须要是闭凸集})



\subsection{仿射函数}
仿射变换和逆仿射变换都是保凸的
\begin{definition}{仿射变换}
    若$f(x)=AX+b$,其中$A \in \R^{m \times n}, b\in \R^m$
    则称$f: \R^n \rightarrow \R^m$是仿射的
\end{definition}
\begin{itemize}
    \item 若$S \in \R^n$是凸集, $f: \R^n \rightarrow \R^m$是仿射的, 则$S$在$f$下的映射\\$f(S)=\{f(x)|x \in S\}$也是凸集
    \item 若$C \in \R^m$是凸集, $g: \R^n \rightarrow \R^m$是仿射的, 则$C$在$g$下的逆映射\\$g^{-1}(C)=\{x|g(x) \in C\}$也是凸集
\end{itemize}


\subsection{线性分式及透视函数}


\section{凸函数}


\end{document}