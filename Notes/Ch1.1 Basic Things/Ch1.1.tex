\documentclass{mytemplate}
\usepackage{hyperref}
\title{Chapter 1.1 基础知识}

\author{%
    21307099\\
    \texttt{liyj323@mail2.sysu.edu.cn} \\
}

\begin{document}
\maketitle

\tableofcontents

\newpage
\section{绪论}
\subsection{最优化}
一般描述为:
\begin{align}
    \text{miminize}  \ \  & f(x)         \\
    \text{subject to}\ \  & x \in \Omega
\end{align}
\subsection{课程内容}
\begin{itemize}
    \item 凸集、凸函数、凸优化问题
    \item 对偶理论
    \item 无约束优化:算法结构、一阶方法、二阶方法
    \item 有约束优化:罚方法、增广拉格朗日乘子法、交替方向乘子法
    \item \textcolor{red}{现代优化算法}
\end{itemize}

\section{基础知识}
\subsection{向量}
\begin{definition}{范数}
    称一个从向量空间$\mathbb{R}^n$到实数域$\mathbb{R}$的非负函数
    $\|\cdot\|$为范数, 如果
    \begin{itemize}
        \item
        \item
        \item 正定性 $\|v\| \geq 0, \forall v \in \mathbb{R}\ \  and \ \  \|v\| = 0 \Leftrightarrow v=0$
        \item 齐次性 $\|\alpha v\| = |\alpha|\|v\|, \forall v \in \mathbb{R}^n, \alpha \in \mathbb{R}$
        \item 三角不等式 $\|v+w\| \leq \|v\|+\|w\|, \forall v,w \in \mathbb{R}^n$
    \end{itemize}
\end{definition}
\begin{definition}{相容范数}
    \\ \vspace*{1em}\hspace*{2em} 满足相容性(次可乘性)的范数,即$\|AB\| \leq \|A\| \cdot \|B\| $
    \\\hspace*{2em} 注意到AB,A,B可能在不同的空间上,相容范数应在所有空间上都满足该性质.
\end{definition}
对于给定$v, w$:
\begin{itemize}
    \item 内积:$\langle x,y\rangle = x^Ty=\sum_{i=1}^{n}x_i y_i$
    \item $\ell_{\infty}=\max_{1\leq i \leq n} |x_i|$
    \item $\ell_{p}(p \geq 1): \|x\|_p = \left(\sum_{i=1}^{n}|x_i|^p\right)^{\frac{1}{p}}$
    \item $\ell_0\text{半范}:\|x\|=(x\text{的所有分量中非0元素的个数})$
\end{itemize}

\begin{proposition}
    $$
        \forall x \in \mathbb{R}:\\
        \|x\|_\infty \leq \|x\|_2\leq\|x\|_1\leq n\|x\|_\infty
    $$
    \text{Cauchy-Schwarz不等式:}
    $$
        \forall x,y \in \mathbb{R}^n:
        -\|x\|_2\|y\|_2\leq \langle  x, y \rangle \leq \|x\|_2 \|y\|_2
    $$
    \text{Hölder不等式:}
    $$
        |\langle x, y \rangle| \leq \|x\|_p \|y\|_q, \ \
        \frac{1}{p} + \frac{1}{q} = 1
    $$
    \noindent
    \text{Minkowski不等式:}
    $$
        \forall x, y\in \mathbb{R}^n, p\in \left[1, \infty \right)
    $$
    \[
        \|x+y\|_p \leq \|x\|_p + \|y\|_q
    \]
\end{proposition}

\subsection{矩阵}
$A\in \mathbb{R}^{m \times n}, B \in \mathbb{R}^{m \times n}$
\begin{definition}{矩阵}
    \begin{itemize}
        \item
        \item 内积: $\langle A, B \rangle = \text{tr} (AB^T) \text{即对应位置元素积的和}$
        \item  Frobenius范数: $\|A\| = \sqrt{\sum_{i,j}|a_{ij}|^2} = \sqrt{\text{tr}(A^T A)}$
        \item 诱导范数\\
              我们希望$\|AX\| \leq \|A\| \|x\|$,故定义
              \[
                  \|A\| \triangleq \sup_{\|x\|=1} \|Ax\| = \sup_{\|x\|\neq0} \frac{\|Ax\|}{\|x\|}
              \]
        \item 诱导$\ell_1$范数(最大列和范数)
              \[
                  \|A\|_1 = \max_j \sum_{i}|a_{ij}|
              \]
        \item 诱导$\ell_2$范数(谱范数)
              \[
                  \|A\|_2 = \sqrt{\lambda_{\max}(A^T A)}
              \]
              特别地,当A是Hermi
        \item 诱导$\ell_{\infty}$范数(最大行和范数)
              \[
                  \|A\|_1 = \max_i \sum_{j}|a_{ij}|
              \]
        \item 核范数
              \[
                  \|A\|_* = \sum_{i=1}^{r} \sigma_i, r=rank{A},\sigma\text{为非零奇异值}
              \]
    \end{itemize}
\end{definition}

\begin{definition}
    正定矩阵\\
    如果对于任意$x \neq 0, x^T Ax >0$,则称A是正定的,记为$A \succ 0$
\end{definition}
\begin{definition}
    半正定矩阵\\
    如果对于任意$x \neq 0, x^T Ax \geq 0$,则称A是半正定的,记为$A \succeq  0$
\end{definition}
\begin{align}
                & \forall x \neq 0, x^T Ax\geq 0                   \\
    \Rightarrow & \forall A\vec{v} =\lambda \vec{v}, v^T A v\geq 0 \\
    \Rightarrow & \lambda v^T v \geq 0                             \\
                & Since\ v^T v \geq 0                              \\
    \Rightarrow & \lambda \geq 0
\end{align}

\subsection{Basic Concepts In Optimization}
\begin{definition}
    最优化问题的数学模型\\
    \begin{align}
        \text{miminize}  \ \  & f(x)         \\
        \text{subject to}\ \  & x \in \Omega
    \end{align}
    \begin{itemize}
        \item 决策变量 $x=(x_1, \dots, x_n)^T$\\
        \item \(f: \Omega \rightarrow \mathbb{R}\)\\
        \item 可行域(决策集) \( \Omega \subseteq \R^n\)
              $\begin{cases}
                      \text{无约束优化问题}, & \Omega = \mathbb{R}^n, \\
                      \text{约束优化问题},  & \text{Otherwise.}
                  \end{cases}
              $
    \end{itemize}
    可行域:等式约束,不等式约束
    \[
        \Omega = \{ x \in \R^n: h(x) =0, c(x) \leq 0 \}
    \]
\end{definition}

\begin{definition}
    \mbox{}\\
    \vspace{-\baselineskip}
    \begin{itemize}
        \item 可行域中的点,或满足约束条件的点称为可行点;
        \item 对于 \(x^* \in \Omega\),若对任意的 \(x \in \Omega\),都有 \(f(x^*) \leq f(x)\),则称 \(x^*\) 是问题的一个全局最优解。对应的目标函数值,即 \(f(x^*)\),为全局最优值。记为 \(x^* \in \arg\min_{x\in\Omega} f(x)\)。
        \item 对于局部最优,若 \(x^*\) 还满足对任意的 \(x \in \Omega\setminus\{x^*\}\),都有 \(f(x) > f(x^*)\),则称 \(x^*\) 为严格全局最优解;
        \item 对于 \(x^* \in \Omega\),若存在 \(x^*\) 的邻域 \(B(x^*, \delta) = \{x: \|x - x^*\| \leq \delta\}\) 使得对于任意的 \(x \in \Omega \cap B(x^*, \delta)\),有 \(f(x^*) \leq f(x)\),则称 \(x^*\) 为局部最优解;
        \item 对于局部最优,若 \(x^*\) 满足对任意的 \(x \in \Omega \cap B(x^*, \delta), x \neq x^*\),有 \(f(x) > f(x^*)\),则称 \(x^*\) 为严格局部最优解。
    \end{itemize}
\end{definition}

\subsection{迭代算法}
基本框架
\begin{itemize}
    \item 取初始点$x^0 \in \mathbb{R}^n$及其他有关参数, $k=0$.
          \\
          \tikz[remember picture] \node[coordinate] (n1) {};\item 验证\textcolor{red}{停机准则}.
    \item 给出搜索方向$d^k\in \mathbb{R}^n$, 通常要求$d^k$是下降方向, 即$(\nabla f_{0} (x^{k}))^T d^{k} < 0$.
    \item 计算迭代步长 $\alpha _k >0$ 使得 $f_0 (x^k + \alpha_k  d^k) < f_0(x^k)$.
    \item  $x^{k+1} := x^k + \alpha_k  d^k, k:=k+1$.\\
          \tikz[remember picture] \node[coordinate] (n2) {};
\end{itemize}

\begin{tikzpicture}[remember picture, overlay]
    \draw[->,thick] (n2) to[bend left=50] (n1);
\end{tikzpicture}

\subsection{最优化问题}
\subsubsection{分类}
\begin{itemize}
    \item 约束优化\&非约束优化
    \item 连续优化\&非连续优化
          \[
              \Omega = \{ x \in \R^n: h(x) =0, c(x) \leq 0 \}
          \]
    \item 线性规划:$f,h,c$都是线性的
    \item 非线性规划:$f,h,c$至少有一个非线性的
    \item 二次规划:$f$是二次函数,$h,c$是线性
    \item 凸优化:目标函数为凸函数,可行域为凸集
\end{itemize}

\subsubsection{算法评价标准}
\paragraph*{全局收敛与局部收敛}
\begin{itemize}
    \item 若$\lim_{k \rightarrow \infty} \|x^k - x^*\| = 0$,则称${x^k}$收敛于$x^*$
    \item 子列收敛于$x^*$,称聚点
    \item 全局收敛性、局部收敛性
\end{itemize}

\paragraph*{收敛速度}
\paragraph{1} 设算法产生的迭代点列${x^k}$收敛于$x^*$,即$x^k \rightarrow x^*$ \\
\[
    \lim_{k \rightarrow \infty} \frac{\|x^{k+!} - x^*\|}{\|x^{k} - x^*\|} = \mu
\]
\begin{itemize}
    \item $\mu \in (0, 1)$称Q-线性收敛(Q表示quotient(分式),线性表示误差取对数后,随迭代步数k显线性)
    \item $\mu = 0$:  Q-超线性收敛,  $\mu = 1$:  Q-次线性收敛
\end{itemize}
\paragraph{2}
\[
    \lim_{k \rightarrow \infty} \frac{\|x^{k+!} - x^*\|}{\|x^{k} - x^*\|^p} = \mu , p > 1 \ \ \text{   称为Q-p阶收敛到}x^*
\]
\paragraph{3}
R-线性收敛, R-超线性收敛                                                       

\end{document}