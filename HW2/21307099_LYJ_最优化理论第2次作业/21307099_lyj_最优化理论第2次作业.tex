\documentclass{article}

\usepackage[utf8]{inputenc}
\usepackage[T1]{fontenc}
\usepackage{ctex}
\usepackage{hyperref}
\usepackage{url}
\usepackage{booktabs}
\usepackage{amsfonts}
\usepackage{amssymb}
\usepackage{amsmath}
\usepackage{amsthm}

\usepackage{algobox}
\usepackage{graphicx}
\usepackage{xcolor}
\usepackage{booktabs}
\usepackage{nicefrac}
\usepackage{microtype}

\usepackage[preprint]{neurips_2020} % Use PDFLaTex to compile
\title{最优化理论 Homework2}

\author{21307099 李英骏}

\begin{document}
\maketitle
\section*{Problem 1}
\begin{align*}
    f(\theta \vec{x} + (1-\theta)\vec{y}) & = \max_{i=1, \ldots, n} f_i(\theta \vec{x} + (1-\theta)\vec{y})                                   \\
                                          & = \max_{i=1, \ldots, n} (\theta f_i(\vec{x}) + (1-\theta)f_i(\vec{y}))                            \\
                                          & \leq \max_{i=1, \ldots, n} (\theta f_i(\vec{x})) + \max_{i=1, \ldots, n} ((1-\theta)f_i(\vec{y})) \\
                                          & = \theta \max_{i=1, \ldots, n} f_i(\vec{x}) + (1-\theta) \max_{i=1, \ldots, n} f_i(\vec{y})       \\
                                          & = \theta f(\vec{x}) + (1-\theta)f(\vec{y})                                                        \\
                                          & \emph{因此f为凸函数}
\end{align*}

\section*{Problem 2}
定义拉格朗日函数为:
\begin{equation*}
    L(x, \lambda, v) = c^T x + \lambda(Gx-h) + v(Ax-b) = (c^T + \lambda G + v A)x - \lambda h - v b
\end{equation*}

\begin{equation*}
    \max_{x \in D} L(x,\lambda,v) =
    \begin{cases}
        c^Tx,    & x \in D    \\
        -\infty, & x \notin D
    \end{cases}
\end{equation*}

\begin{equation*}
    \therefore g(\lambda,v) = \inf_{x \in D} L(x,\lambda,v) =
    \begin{cases}
        -\lambda h - v b, & c^T + \lambda G + v A = 0 \\
        -\infty,          & \text{其他}
    \end{cases}
\end{equation*}

因此对偶问题为:
\begin{equation*}
    \max \{-\lambda h - v b\}
\end{equation*}
\begin{equation*}
    \text{s.t.} \quad c^T + \lambda G + v A = 0
\end{equation*}
\begin{equation*}
    \lambda \geq 0
\end{equation*}

根据强对偶性,有:
\begin{equation*}
    d^* = \sup_{\lambda \geq 0, v} \inf_{x \in D} L(x, \lambda, v)
\end{equation*}
\begin{equation*}
    = \inf_{x \in D} L(x, \lambda^*, v^*) \leq L(x^*, \lambda^*, v^*)
\end{equation*}
\begin{equation*}
    \leq f(x^*) = p^*
\end{equation*}

依据KKT条件:
\begin{equation*}
    \nabla_x L(x^*, \lambda^*, v^*) = c^T + \lambda^* G + v^* A = 0
\end{equation*}
\begin{equation*}
    \therefore \begin{cases}
        Gx^* \leq h            \\
        Ax^* = b               \\
        \lambda^* \geq 0       \\
        \lambda^* (Gx^*-h) = 0 \\
        c^T + \lambda^* G + v^* A = 0
    \end{cases}
\end{equation*}

\section*{Problem 3}

对于给定的参考点 $x_0$ ,我们希望找到最小化目标函数的向量 $x$\\
考虑以下拉格朗日函数 $\mathcal{L}$
\begin{equation}
    \mathcal{L}(\mathbf{x}, \{\mathbf{y}_i\}_{i=1}^N) = \sum_{i=1}^{N} \|\mathbf{y}_i\|_2 + \frac{1}{2}\|\mathbf{x} - \mathbf{x_0}\|_2^2 - \sum_{i=1}^{N} (\mathbf{y}_i - \mathbf{A}_i\mathbf{x} - \mathbf{b}_i)^T\mathbf{z}_i
\end{equation}

为解决这个优化问题,我们首先关注中间变量 $\mathbf{y}_i$,注意到它非光滑
\begin{equation}
    \inf_{\mathbf{y}_i} \left( \|\mathbf{y}_i\|_2^2 + (\mathbf{z}_i)^T\mathbf{y}_i \right) =
    \begin{cases}
        0,       & \text{若 } \|\mathbf{z}_i\|_2 \leq 1, \\
        -\infty, & \text{若 } \|\mathbf{z}_i\|_2 > 1.
    \end{cases}
\end{equation}

接下来需要找到原始变量 $\mathbf{x}$ 的最优值。将 $\mathbf{x}$ 的梯度设为0

\begin{equation}
    \mathbf{x}^* = \mathbf{x_0} + \sum_{i=1}^{N} (\mathbf{A}_i)^T \mathbf{z}_i
\end{equation}

将结果代入拉格朗日函数,得到对偶问题的目标函数。函数描述了在拉格朗日乘子的约束下的最佳值

\begin{equation}
    \mathcal{G}(\mathbf{z}_1, \ldots, \mathbf{z}_N) =
    \begin{cases}
        -\frac{1}{2} \sum_{i=1}^{N} \|\mathbf{A}_i^T \mathbf{z}_i \|_2^2 + \sum_{i=1}^{N} (\mathbf{A}_i \mathbf{x_0} + \mathbf{b}_i)^T \mathbf{z}_i, & \text{如果 } \|\mathbf{z}_i\|_2 \leq 1 \quad \forall i, \\
        -\infty,                                                                                                                                     & \text{其他}.
    \end{cases}
\end{equation}

对偶问题涉及最大化上述对偶函数,并受每个拉格朗日乘子范数必须小于或等于1的约束。

\begin{align}
    \text{maximize} \quad   & \sum_{i=1}^{N} (A_i x_0 + b_i)^T z_i - \frac{1}{2} \sum_{i=1}^{N} \| A_i^T z_i \|_2^2 \\
    \text{subject to} \quad & \|z_i\|_2 \leq 1, \, \forall i \in \{1, \ldots, N\}.
\end{align}

\end{document}